\documentclass[12pt, a4paper]{article}

% Básicos de TeX
\usepackage[portuguese]{babel}
\usepackage{graphicx}
\usepackage{setspace}
\usepackage{caption}
\usepackage{float}
\usepackage{indentfirst}
\setlength{\parindent}{1cm}
\usepackage{hyperref}
\hypersetup{
    colorlinks=true,
    urlcolor=blue,
    linkcolor=blue,
    citecolor=blue,
    pdfborder={0 0 0},
    pdfstartview=FitH,
    hidelinks       
}
\usepackage[ 
    left=2.25cm,
    right=2.25cm,
    top=2.5cm,
    bottom=2.5cm
    ]{geometry}

% Configurações de Fonte

\usepackage{fontspec}
% \usepackage{unicode-math}
\setmainfont{Times New Roman}

% Configurações de ambientes nativos

\newcommand{\secao}[1]{%
    \section*{\normalsize\textbf{#1}}%
}
\usepackage{titlesec}   
\titlespacing*{\section}{0pt}{12pt}{0pt} 

\usepackage{caption}
\captionsetup[figure]{labelfont=bf, labelsep=period, justification=centering, singlelinecheck=false} 

\captionsetup[table]{labelsep=period, labelfont=bf, justification=centering, singlelinecheck=false}

\usepackage{fancyhdr}
\usepackage{etoolbox}

\pagestyle{fancy}
\fancyhf{} % Limpa tudo
\renewcommand{\headrulewidth}{0pt} % Remove linha do cabeçalho

% CABEÇALHO
\fancyhead[L]{\footnotesize Prenome Nome Sobrenome, Prenome Nome Sobrenome, Prenome Nome Sobrenome}
\fancyhead[R]{\thepage}

% RODAPÉ    
\fancyfoot[R]{\footnotesize \textbf{Semana de Educação e Pesquisa em Matemática da UECE}
\\ ISBN: 2317-6458}



\begin{document}
\thispagestyle{plain}

\begin{figure}[H]
    \centering
    \includegraphics[width=5.6 in, height=1.4 in]{lib/logoSepmat.png}

\end{figure}

% Título
\begin{center}
    \textbf{\large Instruções e Modelo para Autores de Resumos Expandidos da Semana de Educação e Pesquisa em Matemática - SEPMAT}
    
    \vspace{0.5cm}
    
    % Autores
    \textbf{Prenome Nome Sobrenome\textsuperscript{1}, Prenome Nome Sobrenome\textsuperscript{2},}
    
    \textbf{Prenome Nome Sobrenome\textsuperscript{3}, Prenome Nome Sobrenome\textsuperscript{4}}
    
    \textbf{Prenome Nome Sobrenome\textsuperscript{5}, Prenome Nome Sobrenome\textsuperscript{6}}
    
    % \vspace{0.3cm}
    
    % Instituições
    \small
    \textsuperscript{1}Universidade Estadual do Ceará, Nome do Centro/Faculdade, e-mail:
    
    \textsuperscript{2}Universidade Estadual do Ceará, Nome do Centro/Faculdade, e-mail:
    
    \textsuperscript{3}Universidade Estadual do Ceará, Nome do Centro/Faculdade, e-mail:
    
    \textsuperscript{4}Universidade Estadual do Ceará, Nome do Centro/Faculdade, e-mail:
    
    \textsuperscript{5}Universidade Estadual do Ceará, Nome do Centro/Faculdade, e-mail:
    
    \textsuperscript{6}Universidade Estadual do Ceará, Nome do Centro/Faculdade, e-mail: professor@uece.br
\end{center}

% Resumo
% \vspace{12pt}
\begin{quote}
    % \setlength{\leftskip}{0.1 cm}
    % \setlength{\rightskip}{0.8 cm}
    
    \textbf{RESUMO.} Este metaresumo descreve o estilo a ser usado na confecção de resumo expandido para publicação nos anais da VIII Semana de Educação e Pesquisa em Matemática - SEPMAT. O resumo expandido deverá conter as seguintes seções: RESUMO, INTRODUÇÃO, METODOLOGIA, RESULTADOS E DISCUSSÃO, CONSIDERAÇÕES FINAIS e REFERÊNCIAS. A fonte deve ser Times New Roman, o espaçamento entre linhas deve ser simples, e o parágrafo deve ser justificado. O resumo expandido deve ter um mínimo de 03 (três) e um máximo de 04 (quatro) laudas. Este metaresumo deve ser baixado e usado pelo autor como base para a escrita de seu texto.
    
    Palavras-chave: Primeira. Segunda. Terceira.
\end{quote}

\vspace*{0.2cm}

% Seção 1
\secao{\textbf{1. INFORMAÇÕES GERAIS}}

Os resumos expandidos devem ser escritos em português. O texto deve ser escrito em coluna única com margens superior, inferior, esquerda e direita de 2,5 cm. A fonte principal deve ser Times New Roman, tamanho de 12 pontos. O resumo expandido deve ter um mínimo de 03 (três) e um máximo de 04 (quatro) páginas. O cabeçalho e o rodapé não podem ser alterados.

% Seção 2
\secao{\textbf{2. PRIMEIRA PÁGINA}}
A primeira página deve conter na sua parte superior o título do trabalho, o nome, a instituição e o endereço eletrônico dos autores. O título deve ser centralizado, em fonte estilo negrito, tamanho de 14 pontos. Os nomes dos/as autores/as (no máximo seis) devem ser centralizados, separados por vírgula, em tamanho de 12 pontos, negrito e espaço simples. Os nomes das instituições e os endereços eletrônicos devem ser centralizados, em fonte de 11 pontos. Após esses dados virá o resumo em fonte Times New Roman, tamanho de 12 pontos, recuado 0.8 cm em ambos os lados. A palavra \textbf{RESUMO} deve ser escrita em letra maiúscula, em negrito e deve preceder o texto que não pode ultrapassar 10 linhas. No final do resumo o(s) autor(es) deve(m) indicar 03 (três) palavras-chave, separadas por ponto. Em seguida, os autores podem iniciar a(s) seção (ões) subsequentes.

% Seção 3
\secao{\textbf{2. PRIMEIRA PÁGINA}}
A primeira página deve conter na sua parte superior o título do trabalho, o nome, a instituição e o endereço eletrônico dos autores. O título deve ser centralizado, em fonte estilo negrito, tamanho de 14 pontos. Os nomes dos/as autores/as (no máximo seis) devem ser centralizados, separados por vírgula, em tamanho de 12 pontos, negrito e espaço simples. Os nomes das instituições e os endereços eletrônicos devem ser centralizados, em fonte de 11 pontos. Após esses dados virá o resumo em fonte Times New Roman, tamanho de 12 pontos, recuado 0.8 cm em ambos os lados. A palavra \textbf{RESUMO} deve ser escrita em letra maiúscula, em negrito e deve preceder o texto que não pode ultrapassar 10 linhas. No final do resumo o(s) autor(es) deve(m) indicar 03 (três) palavras-chave, separadas por ponto. Em seguida, os autores podem iniciar a(s) seção (ões) subsequentes.

% Seção 4
\secao{\textbf{4. REFERÊNCIAS}}
As referências deverão ser apresentadas de acordo com a Norma ABNT NBR 6023. Consultar o Guia de Normalização de Trabalhos Acadêmicos da UECE no link 

\url{https://www.uece.br/biblioteca/wp-content/uploads/sites/27/2024/09/GUIA-UECE-2024-Atualizado-1.pdf}

\vspace{1cm}

\noindent
\textbf{OBSERVAÇÃO:} Este template deve ser utilizado para elaboração do resumo expandido, mas o arquivo enviado deve ser em formato PDF.

\end{document}