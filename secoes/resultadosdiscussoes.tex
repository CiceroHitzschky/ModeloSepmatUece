\secao{\textbf{3. SEÇÕES E PARÁGRAFOS}}
Os títulos das seções (INTRODUÇÃO, METODOLOGIA, RESULTADOS E DISCUSSÃO, CONSIDERAÇÕES FINAIS e REFERÊNCIAS) devem ser numerados, alinhados à esquerda, escritos em letra maiúscula, negrito, fonte Times New Roman, tamanho de 12 pontos. A numeração da seção é obrigatória. Os parágrafos de cada seção devem ser recuados em 1,25 cm, com espaçamento simples entre linhas e entre parágrafos.

As Figuras (gráficos, fotos, desenhos, mapas, fluxograma, etc) devem ser utilizadas com parcimônia e possuir resolução de até 300 dpi.

\begin{figure}[H]
    \centering
    \fbox{\includegraphics[width=6.1cm, height=4.9cm]{lib/eximg.png}}

    \caption{\textbf{"Não, você não foi baixado da Internet. Você nasceu".}}
\end{figure}

A legenda da figura deve ser colocada abaixo da figura e a legenda da tabela e do quadro deve ser colocada antes da tabela e do quadro. A fonte usada nas figuras, tabelas e quadro deve ser Times New Roman, negrito, de tamanho de 10 pontos.

As legendas de figuras, tabelas e quadros, independentemente do número de linhas, devem ser centralizadas, como pode ser visto nas Figura 1 e Tabela 1, escritas em fonte Times New Roman, 10 pontos e negrito.

\begin{table}[H]
    \centering
    \caption{\textbf{Número de trabalhos apresentados na XXVII Semana Universitária da UECE por área de conhecimento.}}
    \begin{tabular}{|l|c|}
        \hline
        \textbf{ÁREA DE CONHECIMENTO} & \textbf{TOTAL} \\
        \hline
        Ciências Exatas e da Terra & 272 \\
        \hline
        Ciências Biológicas & 160 \\
        \hline
        Ciências da Saúde & 869 \\
        \hline
        Ciências Agrárias & 141 \\
        \hline
        Ciências Sociais Aplicadas & 169 \\
        \hline
        Ciências Humanas & 675 \\
        \hline
        Linguística, Letras e Artes & 198 \\
        \hline
        \textbf{TOTAL} & \textbf{2.527} \\
        \hline
    \end{tabular}
\end{table}

Nas tabelas e quadros, evite o uso dos fundos coloridos ou sombreados, linhas mais densas, linhas duplas e molduras desnecessárias. Ao relatar dados empíricos, use apenas dígitos decimais que garantam sua precisão e reprodutibilidade.